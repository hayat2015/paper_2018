%%%%%%%%%%%%%%%%%%%%%%% file typeinst.tex %%%%%%%%%%%%%%%%%%%%%%%%%
%
% This is the LaTeX source for the instructions to authors using
% the LaTeX document class 'llncs.cls' for contributions to
% the Lecture Notes in Computer Sciences series.
% http://www.springer.com/lncs       Springer Heidelberg 2006/05/04
%
% It may be used as a template for your own input - copy it
% to a new file with a new name and use it as the basis
% for your article.
%
% NB: the document class 'llncs' has its own and detailed documentation, see
% ftp://ftp.springer.de/data/pubftp/pub/tex/latex/llncs/latex2e/llncsdoc.pdf
%
%%%%%%%%%%%%%%%%%%%%%%%%%%%%%%%%%%%%%%%%%%%%%%%%%%%%%%%%%%%%%%%%%%%


\documentclass[runningheads,a4paper]{llncs}

\usepackage{amssymb}
\setcounter{tocdepth}{3}
\usepackage{graphicx}
\usepackage{url}
%\usepackage{url}
%\urldef{\mailsa}\path|{sbendoukha, .haas, frank.holzwarth,|
%\urldef{\mailsb}\path|anna.kramer, leonie.kunz, christine.reiss, nicole.sator,|
%\urldef{\mailsc}\path|erika.siebert-cole, peter.strasser, lncs}@springer.com|    
\newcommand{\keywords}[1]{\par\addvspace\baselineskip
\noindent\keywordname\enspace\ignorespaces#1}
\input{./newcommands}
\begin{document}

\mainmatter  % start of an individual contribution

% first the title is needed
\title{On the Use of Reference Nets for Building Cloud-based Scientific Workflows}

% a short form should be given in case it is too long for the running head
\titlerunning{Cloud-based Image Processing}

% the name(s) of the author(s) follow(s) next
%
% NB: Chinese authors should write their first names(s) in front of
% their surnames. This ensures that the names appear correctly in
% the running heads and the author index.
%
\author{Hayat Bendoukha\inst{1}%
%\thanks{Please note that the LNCS Editorial assumes that all authors have used
%the western naming convention, with given names preceding surnames. This determines
%the structure of the names in the running heads and the author index.}%
\and Sofiane Bendoukha\inst{2}\and Yahya Slimani \inst{3}}
%
\authorrunning{Hayat Bendoukha et al.}
% (feature abused for this document to repeat the title also on left hand pages)

% the affiliations are given next; don't give your e-mail address
% unless you accept that it will be published
\institute{Department of Computer Science\\
Faculty of Mathematics and Computer Science\\
Universit\'e des Sciences et de la Technologie d'Oran Mohamed Boudiaf, USTO-MB, BP 1505, El M'naouer,  31000  Oran Alg\'erie\\
\texttt{bendoukhayat@univ-usto.dz}
\and
Theoretical Foundations of Computer Science (TGI)\\
Department of Informatics, University of Hamburg, Germany\\
\texttt{sbendoukha@informatik.uni-hamburg.de}\\
\and
Département Informatique
Institut Sup\'erieur des Arts Multim\'edia (ISAMM)
Universit\'e de la Manouba
 \\
Campus Universitaire, 2010 Manouba, Tunisie\\
\texttt{yahya.slimani@fst.rnu.tn}
}
 
%
% NB: a more complex sample for affiliations and the mapping to the
% corresponding authors can be found in the file "llncs.dem"
% (search for the string "\mainmatter" where a contribution starts).
% "llncs.dem" accompanies the document class "llncs.cls".
%

\toctitle{Lecture Notes in Computer Science}
\tocauthor{Sofiane Bendoukha, Hayat Bendoukha, Daniel Moldt}
\maketitle


\begin{abstract}
Cloud computing provides scientists with a large number of powerful resources.
%
These resources enhance the productivity of the whole system.
%
Running complex scientific workflows on Cloud resources rather then on-premise increases the performance of execution.
%
Nevertheless, traditional scientific workflow management (SWfM) systems are not yet adapted for the Cloud.
%
Creation of scientific workflows and their execution in the Cloud still face many challenges due to the complexity of the Cloud environment. 
%
Migrating a part or the whole scientific application to the Cloud is not trivial .
%
It should be based on a solid strategy.
%
Questions like: Where to store the data and where to execute the processes need to be investigated.
%
In this paper, we first present \RenewGrass{}, a tool for modeling and executing image processing workflows by \emph{reference nets}.
%
Then, we discuss the deployment of \RenewGrass{} into the Cloud.
%
This work includes also a use case example related to the remote sensing domain.
%
%Here we should emphasize the results provided in this paper.


\keywords{Workflows, \RenewGrass, Petri Nets, Cloud computing, Image processing}
\end{abstract}


\section{Introduction}
\input{introduction}

\section{Related Work and Background}
\input{functionality}

\section{\RenewGrass{}}
\input{grassintegration}

\section{\RenewGrass{} in the Cloud}
\input{cloudmigration}


\section{Discussion}
\input{discussion}

\section{Conclusion}
\input{conclusion}




\bibliography{FMi_2015_AISC}

\bibliographystyle{plain}
\end{document}
