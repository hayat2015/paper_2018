\label{sec:relatedwork}
%
A model is a simplified representation of a phenomenon or a system.
%
Models in the remote sensing domain are important to analyze and to understand the behavior of the whole processing steps.
%
There are different kind of models, the complete classification.
%
Their presentation and explanation are out of scope of this paper.
%
We only notice that the models are classified by purpose, methodology and logic.
%
In this work we focus on \textit{Process Models}, which integrate existing knowledge into a set of relationships and equations for quantifying the physical process.
%
These models are typically raster-based.
%
Most of GIS (commercial or open source) provide tools and assist scientists creating their models created by dragging and dropping tools. 
%
In most of established GIS softwares, the tool is named \textit{ModelBuilder}.
%
A ModelBuilder is a graphical user interface that helps creating models. 
%
In ModelBuilder, a process is composed of input data, the tool to apply to the data and the resulting output data. 
%
Famous existing ModelBuilders are the ArcGIS (\cite{Armstrong09}) and the \textsc{Grass} ModelBuilder\footnote{\url{http://grasswiki.osgeo.org/wiki/WxGUI_Graphical_Modeler}}.
%
In ArcGIS, the model is represented by a flux diagram in a graphic user interface that facilitates to create, visualize, edit, and execute geoprocessing workflows, to use and reuse them, to share and apply them to different geographic areas.
%
In the other side, \textsc{Grass} GIS provides also commands for preparing input variables, running and viewing the model.    
%

The novelty of this work is that we use Petri nets as modeling techniques in contrast to the tools mentioned above.
%
Petri nets provide both graphical notation and a precise mathematical notions.
%
The first feature offers a clear way to specify complex systems, while the second feature permits to analyze the created model.
%
We mean by analysis, the possibility to verify certain properties of Peri nets models such as the \emph{reachability} of certain markings, \emph{boundedness} of places, \emph{liveness} and deadlock-freedom.

%
Concerning this issue, \Renew{} already provides several plug-ins that help to analyze the modeled workflows.
%
For example in \cite{Hewelt+11}, the authors investigate the integration of an external verification tool called LoLA (Low Level Net Analyser, \cite{Karsten+00}).
%
LoLa is a verification tool for place/transition and colored nets.
%
In \cite{Cabac+11}, the authors introduce net components that are designed to test the behavior of Petri nets under development.
%
In a second step, the execution of these test components is automated and integrated into the unit testing framework JUnit. 
%

\section{\RenewGrass{} in \Renew{}}
\label{sec:renewgrassinrenew}
%
Although \RenewGrass{} does not require an additional graphical user interface for the modeling and the simulation of workflows, we have implemented front-end functionalities.
%
An entry is added for \RenewGrass{} in the menu. 
%
Users have the possibility to open predefined image processing models or to activate the Grass Net Components (see Section~\ref{sec:grassnetcomponents}). 
%For instance, Fig. \ref{fig:netcomp} shows the Grass Nets Components, which can be selected from the \Renew{} palette.
%
Furthermore, to use the Grass Net Components, a new palette is added to the GUI of \Renew{} (see Fig.~\ref{fig:netcomp}).  
%
\begin{figure}[!t]
    \centering
  \includegraphics[width=0.45\textwidth]{images/renewgrassnets}
\caption{The Grass Net Components}
\label{fig:netcomp}
\end{figure} 

%In order to be integrated into \Renew{}, \RenewGrass{} is written in Java.
%
%The Grass GIS modules (see section \ref{sec:grassintegration}) can be directly invoked from the Petri nets transitions. 
%

%
%The integration of \RenewGrass{} in \Renew{} offers many advantages:
%

\begin{figure}[!t]
    \centering
  \includegraphics[width=0.45\textwidth,height=0.10\textheight]{images/screenshotMenu}
\caption{\RenewGrass{} in \Renew{} Menu}
\label{fig:grassmenu}
\end{figure}




