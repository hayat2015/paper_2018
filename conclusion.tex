\label{sec:conclusion}
%
The objective of the work presented in this paper is twofold.
%
On the one side, we provide techniques and tools to support scientists building their applications.
%
For this purpose, a geoprocessing tool named \RenewGrass{} is presented.
%
The tool has been implemented and successfully integrated in the modeling and simulation tool \Renew{}.
%the implementation of a geoprocessing tool  for \Renew{}.  
%
%This tool extends \Renew{} to be able to support another kind of workflows (scientific workflows) apart the business workflows. 
%
The application domain of \RenewGrass{} is the remote sensing, especially image processing, a kind of scientific workflows. 
%
Therefore, we afford scientists with a palette of processing functionalities based on the Grass GIS. 
%
Furthermore, we discuss the extension of the current work by the integration of the Cloud technology. 
%
%Although the first objectives are reached, we are actively working to enable the \RenewGrass{} in the Cloud.
For this purpose, we introduce migration patterns and introduced our architecture for the deployment of workflows onto Cloud providers. 
%
The natural next step is to concertize the deployment mechanisms introduced in Section~\ref{sec:Cloudmigration}.
%
This means concretely to implement the functionality of each agent.
%
In our perspective, this can be performed by using the \Mulan{}/\Capa{} framework and following the \Paose{} approach.
